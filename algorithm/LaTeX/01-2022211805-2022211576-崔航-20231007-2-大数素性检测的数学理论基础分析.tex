\documentclass[a4paper]{article}
\usepackage[margin=1in]{geometry} % 设置边距,符合Word设定
\usepackage{ctex}
% Keywords command
\providecommand{\keywords}[1]
{
    \small	
    \textbf{关键词:} #1
}

% Keywords command
\providecommand{\keywords}[1]
{
  \small	
  % \textbf{\textit{Keywords---}} #1
  \textbf{关键词:} #1
}
\title{\heiti\zihao{3} 大数素性检测的数学理论基础分析}
\date{班级:2022211805 学号:2022211576}
\author{\songti 崔航}
\begin{document}
    \maketitle
\begin{abstract}
    对于素数的研究一直是令数学家们着迷,如何对大素数进行准确地检测也是一直以来是密码学的重要研究方向。
    本文介绍了几种素数检测方法,分别是Fermat素数检测,Solovay-Strassen素数检测和Miller-Rabin素数检测等。
    并对这几种方法进行了代码实现。
\end{abstract}
\keywords{素数检测,伪素数,强伪素数}
\tableofcontents
\section{Fermat小定理}
当$p$为素数时,对于任意整数$a$,有$a^p\equiv a\pmod p$。即$a^{p-1}\equiv1\pmod p$。
\par
素数检测的基石定理,十分重要。
\section{伪素数}
在检测素数时可能检测到符合素数特性的合数,在素数检测时需要格外注意。
\subsection{Carmichael伪素数}
由Fermat's Little Theorem可知,若对于任意整数$a$,有$a^p\equiv a\pmod p$,则$p$可能为素数。但是,若$p$为合数,仍有可能满足$a^p\equiv a\pmod p$。此时,$p$被称为Carmichael伪素数。

\subsection{Fermat素数检测}
\begin{enumerate}
    \item 给定奇整数$n\geq3$和安全参数$t$。
    \item 随机选取整数$b$,$(b,n)=1,2\leq b\leq n-2$.
    \item 计算$r=b^{n-1}\pmod n$。
    \item 若$r\neq1$,则$n$为合数。
    \item 若$r=1$,则$n$可能为素数,重复步骤2-4,共重复$t$次。
\end{enumerate}
\subsubsection{算法实现}
\begin{verbatim}
    def fermat(n, k):
        if n == 2:
            return True
        if n % 2 == 0:
            return False
        for i in range(k):
            a = random.randint(2, n - 2)
            if pow(a, n - 1, n) != 1:
                return False
        return True
\end{verbatim}
\subsection{平方因子检测}
判断一个数是否为Carmichael伪素数,只需要判断其是否有平方因子。
设$n$为一个有平方因子的整数,则存在整数$a$,
$(a,n)=1$,$a^{(n-1)}\not \equiv 1\pmod n$。
\subsubsection{算法实现}
\begin{verbatim}
    def square_factor(n):
        for i in range(2, int(n ** 0.5) + 1):
            if n % (i ** 2) == 0:
                return True
        return False
\end{verbatim}
\subsection{Euler伪素数}
设$n$是一个正奇合数,设整数$b$与$n$互素,
若$b^{\frac{n-1}{2}}\equiv(\frac{b}{n})\pmod n$,其中$(\frac{b}{n})$为Jacobi符号,且$(\frac{b}{n})\neq b^{\frac{n-1}{2}}\pmod n$,则$n$被称为Euler伪素数。
则$n$被称为Euler伪素数。
\subsubsection{Solovary-Strassen素数检测}
\begin{enumerate}
    \item 给定奇整数$n\geq3$和安全参数$t$。
    \item 随机选取整数$b$,$(b,n)=1,2\leq b\leq n-2$.
    \item 计算$r=b^{\frac{n-1}{2}}\pmod n$。
    \item 若$r\not\equiv\pm1\pmod n$,则$n$为合数。
    \item 计算Jacobi符号$(\frac{b}{n})$。
    \item 如果$r\not=s $,则$n$为合数。
    \item 若$r\equiv\pm1\pmod n$,则$n$可能为素数,重复步骤2-4,共重复$t$次。
\end{enumerate}
\subsubsection{算法实现}
\begin{verbatim}
    def solovay_strassen(n, k):
        if n == 2:
            return True
        if n % 2 == 0:
            return False
        for i in range(k):
            a = random.randint(2, n - 2)
            if pow(a, (n - 1) // 2, n) != jacobi(a, n) % n:
                return False
        return True
\end{verbatim}
\section{强伪素数}
设一个奇合数$n$,且有表达式$n-1=2^sd$,其中$d$为奇数。若对于任意整数$a$,有$$a^d\equiv1\pmod n$$或$$a^{2^rd}\equiv-1\pmod n$$,则$n$被称为强伪素数。
\subsection{Miller-Rabin素数检测}
\begin{enumerate}
    \item 给定奇整数$n\geq3$和安全参数$k$。
    \item d为奇整数,$n-1=2^sd$。
    \item 随机选取整数$b$,$(b,n)=1,2\leq b\leq n-2$.
    \item 计算$r\equiv b^d\pmod n$。
    \item 若$r=1$或$r=n-1$,则$n$可能为素数,重复步骤2-4,共重复$k$次。
    \item 否则$r_{1}\equiv r^2\pmod n$,若$r_{1}=n-1$,则$n$可能为素数,重复步骤2-4,共重复$k$次。
    \item 否则计算$r_{2}\equiv r_{1}^2\pmod n$,若$r_{2}=n-1$,则$n$可能为素数,重复步骤2-4,共重复$k$次。
    \item 依此类推,直到$r_{t-1}=n-1$,则$n$可能为素数,重复步骤2-4,共重复$k$次。
    \item 若$r_{t-1}\neq n-1$,则$n$为合数。
\end{enumerate}
\subsubsection{算法实现}
\begin{verbatim}
    def miller_rabin(n, k):
        if n == 2:
            return True
        if n % 2 == 0:
            return False
        s = 0
        d = n - 1
        while d % 2 == 0:
            s += 1
            d //= 2
        for i in range(k):
            a = random.randint(2, n - 2)
            x = pow(a, d, n)
            if x == 1 or x == n - 1:
                continue
            for j in range(s - 1):
                x = pow(x, 2, n)
                if x == n - 1:
                    break
            else:
                return False
        return True
\end{verbatim}
\section{总结}
本文介绍了几种素数检测方法,分别是Fermat素数检测,Solovay-Strassen素数检测和Miller-Rabin素数检测。
并对这几种方法进行了代码实现.除了这些实用的素数概率性检测,还有一些更加复杂的素数检测方法,如AKS素数检测,Lenstra素数检测等,安全性和实用性不高,本文不再赘述。
\begin{thebibliography}{1000}  
    
    \bibitem{ref1}刘学军,邢玲玲,林和平,粟浩然.Miller-Rabin素数检测优化算法研究与实现[J].信息技术,2008,32(12):
    \\141-143+147.
    \bibitem{ref2}魏成行. 素性检测算法研究及其在现代密码学中的应用[D].山东大学,2010.
    \bibitem{ref3}彭韬,陈文庆.基于VB的大素数Solovay-Strassen检测的设计与实现[J].电子技术与软件工程,2020,(10):
    \\161-162.
    \bibitem{ref4}陈恭亮.信息安全数学基础[M].清华大学出版社,2004.
\end{thebibliography}
\end{document}
