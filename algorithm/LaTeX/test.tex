\documentclass[a4paper]{article}
\usepackage[margin=1in]{geometry} % 设置边距,符合Word设定
\usepackage{ctex}
\usepackage{lipsum}
\title{\heiti\zihao{3} Report One}
\author{\songti 崔航}
\date{2023.09.30}
\begin{document}
    \maketitle
\begin{abstract}
    \lipsum[2]
\end{abstract}
\tableofcontents
\section{Mille-Rabin方法}
\subsection{费马小定理}
当$p$为素数时,对于任意整数$a$,有$a^p\equiv a\pmod p$。
即$a^{p-1}\equiv 1\pmod p$。
\subsubsection{算法描述}
\begin{enumerate}
    \item 选取一个整数$a$,使得$1<a<n$。
    \item 计算$a^{n-1}\pmod n$。
    \item 若$a^{n-1}\not\equiv 1\pmod n$,则$n$为合数;否则,$n$可能为素数。
    \item 重复步骤1-3,$k$次后,若$n$为合数,则$n$为合数;否则,$n$可能为素数。
\end{enumerate}
\subsubsection{证明}
\begin{enumerate}
    \item 若$n$为素数,由费马小定理可知,$a^{n-1}\equiv 1\pmod n$。
    \item 若$n$为合数,由费马小定理可知,$a^{n-1}\equiv 1\pmod n$。

\end{enumerate}
\subsection{算法实现}
\subsubsection{python实现}
\begin{verbatim}
    def miller_rabin(n, k):
        if n == 2:
            return True
        if n % 2 == 0:
            return False
        s = 0
        d = n - 1
        while d % 2 == 0:
            s += 1
            d //= 2
        for i in range(k):
            a = random.randint(2, n - 1)
            x = pow(a, d, n)
            if x == 1 or x == n - 1:
                continue
            for j in range(s - 1):
                x = pow(x, 2, n)
                if x == 1:
                    return False
                if x == n - 1:
                    break
            else:
                return False
        return True
\end{verbatim}
\section{}

\end{document}