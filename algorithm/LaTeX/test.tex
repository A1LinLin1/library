\documentclass[a4paper]{article}
\usepackage[margin=1in]{geometry} % 设置边距,符合Word设定
\usepackage{ctex}
% Keywords command
\providecommand{\keywords}[1]
{
    \small	
    \textbf{关键词:} #1
}

% Keywords command
\providecommand{\keywords}[1]
{
  \small	
  % \textbf{\textit{Keywords---}} #1
  \textbf{关键词:} #1
}
\title{\heiti\zihao{3} 椭圆曲线公钥密码算法的数学原理分析}
\date{班级:2022211805 学号:2022211576}
\author{\songti 崔航}
\begin{document}
    \maketitle
\begin{abstract}

\end{abstract}
\keywords{素数检测,伪素数,强伪素数}
\tableofcontents
\section{ECC}
椭圆曲线加密算法,简称ECC,是基于椭圆曲线数学理论实现的一种非对称加密算法。相比于RSA,ECC使用更短的密钥,实现与RSA相当或更高的安全,
\section{阿贝尔群}
椭圆曲线的运算是在一个阿贝尔群上进行的,所以我们先来了解一下阿贝尔群的定义。
\subsection{群}
加群是一个集合,集合中的元素可以进行加法运算,且满足以下条件:
\item 1.封闭性:对于任意的$a,b\in G$,有$a+b\in G$。
\item 2.结合律:对于任意的$a,b,c\in G$,有$(a+b)+c=a+(b+c)$。
\item 3.单位元:存在一个元素$0\in G$,使得对于任意的$a\in G$,有$a+0=0+a=a$。
\item 4.逆元:对于任意的$a\in G$,存在一个元素$-a\in G$,使得$a+(-a)=(-a)+a=0$。
阿贝尔群是一个加群,且满足交换律,即对于任意的$a,b\in G$,有$a+b=b+a$。


\subsection{阿贝尔群}



\begin{thebibliography}{1000}  
    
    \bibitem{ref1}
    \bibitem{ref2}
    \bibitem{ref3}
    \bibitem{ref4}
\end{thebibliography}
\end{document}
