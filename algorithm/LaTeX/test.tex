\documentclass[a4paper]{article}
\usepackage[margin=1in]{geometry} % 设置边距,符合Word设定
\usepackage{ctex}
% Keywords command
\providecommand{\keywords}[1]
{
    \small	
    \textbf{关键词:} #1
}

% Keywords command
\providecommand{\keywords}[1]
{
  \small	
  % \textbf{\textit{Keywords---}} #1
  \textbf{关键词:} #1
}
\title{\heiti\zihao{3} 椭圆曲线公钥密码算法的数学原理分析}
\date{班级:2022211805 学号:2022211576}
\author{\songti 崔航}
\begin{document}
    \maketitle
\begin{abstract}

\end{abstract}
\keywords{素数检测,伪素数,强伪素数}
\tableofcontents
\section{Fermat小定理}

\section{总结}
本文介绍了几种素数检测方法,分别是Fermat素数检测,Solovay-Strassen素数检测和Miller-Rabin素数检测。
并对这几种方法进行了代码实现.除了这些实用的素数概率性检测,还有一些更加复杂的素数检测方法,如AKS素数检测,Lenstra素数检测等,安全性和实用性不高,本文不再赘述。
\begin{thebibliography}{1000}  
    
    \bibitem{ref1}刘学军,邢玲玲,林和平,粟浩然.Miller-Rabin素数检测优化算法研究与实现[J].信息技术,2008,32(12):
    \\141-143+147.
    \bibitem{ref2}魏成行. 素性检测算法研究及其在现代密码学中的应用[D].山东大学,2010.
    \bibitem{ref3}彭韬,陈文庆.基于VB的大素数Solovay-Strassen检测的设计与实现[J].电子技术与软件工程,2020,(10):
    \\161-162.
    \bibitem{ref4}陈恭亮.信息安全数学基础[M].清华大学出版社,2004.
\end{thebibliography}
\end{document}
