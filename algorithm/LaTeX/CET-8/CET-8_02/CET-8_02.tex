\documentclass[a4paper]{article}
\usepackage[margin=1in]{geometry} % 设置边距,符合Word设定
\usepackage{ctex}
\usepackage{url}
\usepackage{graphicX}
% Keywords command
\providecommand{\keywords}[1]
{
    \small	
    \textbf{关键词:} #1
}

% Keywords command
\providecommand{\keywords}[1]
{
  \small	
  % \textbf{\textit{Keywords---}} #1
  \textbf{关键词:} #1
}
\title{\heiti\zihao{3} TEC-8实验\_02\_双端口存储器实验}
\date{班级:2022211805  学号:2022211576  姓名:崔航}
% \author{\songti 崔航}
\begin{document}
    \maketitle
% \begin{abstract}


% \end{abstract}
% \keywords{群、椭圆曲线、ECC、加解密}
\tableofcontents
\newpage

\section{实验目的}
\begin{enumerate}
    
    \item 了解双端口静态存储器IDT7132的工作特性及使用方法;
    \item 了解半导体存储器怎样存储和读取数据;
    \item 了解双端口双端口存储器怎样并行读写;
    \item 熟悉TEC-8模型计算机存储器部分的数据通路。
    
\end{enumerate}

\section{实验内容}
\subsection{微程序控制器方式}
\begin{itemize}
    \item 从存储器地址10H开始,通过左端口连续向双端口RAM中写入3个数:85H、60H、38H。在写的过程中,在右端口检测写的数据是否正确。
    \item 从存储器地址10H开始,连续从双端口RAM的左端口和右端口同时读出存储器的内容。
\end{itemize}
\subsection{独立模式}
\begin {itemize}
    \item 将运算器模块与实验台操作板上的线路进行连接。由于运算器模块内部的连线已经由印制电路板连接好,故接线任务仅仅是完成数据开关、控制信号模拟开关与运算器模块的外部连线。
    \item 用开关$K_{15}-K_{0}$向通用寄存器堆RF内的$R_3-R_0$寄存器置入数据。然后读出$R_3-R_0$寄存器的数据,在数据总线DBUS上显示出来。
    \item 验证ALU的正逻辑算术、逻辑运算功能。
    
\end{itemize}
\section{实验过程}
\subsection{微程序控制模式}

\subsection{独立模式}

\section{实验思考与心得}

\end{document}